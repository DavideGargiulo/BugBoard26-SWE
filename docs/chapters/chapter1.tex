\chapter{Specifica dei Requisiti Software}

%Glossario qui

\section{Glossario}

\begin{itemize}
  \item \textbf{Amministratore} \\
        Ruolo utente con privilegi completi nel sistema, può gestire progetti, utenti e tutte le issue indipendentemente dall'assegnazione.
  \\
  \item \textbf{Bug, Feature, Question, Documentation} \\
        Tipologie di issue che possono esistere nel sistema.
  \\
  \item \textbf{Commento} \\
        Testo aggiunto da un utente a una issue esistente per fornire aggiornamenti o discussioni.
  \\
  \item \textbf{Dashboard} \\
        Pagina principale dell'applicazione che mostra statistiche e panoramica delle issue.
  \\
  \item \textbf{Issue} \\
  Entità centrale del sistema che rappresenta una segnalazione (bug, feature, question, documentation).
  \\
  \item \textbf{Priorità} \\
        Attributo di una issue che indica l'urgenza (Alta, Media, Bassa).
  \\
  \item \textbf{Progetto} \\
        Contenitore logico che raggruppa issue correlate.
  \\
  \item \textbf{Sistema} \\
        Insieme degli artefatti software impiegati nella fornitura dei servizi richiesti dal committente.
  \\
  \item \textbf{Stato} \\
        Attributo di una issue che indica la fase del ciclo di vita (TODO, In-Progress, Done).
  \\
  \item \textbf{Use Case (UC)} \\
        Tipo di interazione offerto dal sistema che porta un vantaggio a un utente esterno
  \\
  \item \textbf{Utente Standard} \\
        Ruolo utente con permessi limitati, può agire solo sulle issue a lui assegnate.
\end{itemize}

\newpage

\section{Modellazione di tutti i casi d'uso richiesti tramite Use Case Diagrams}

\begin{figure}[h]
  \centering
  \includegraphics[width=0.50\textwidth]{images/use_case_diagram.pdf}
\end{figure}