\chapter{Individuazione e caratterizzazione del target degli utenti,
tramite Personas}

\section{Personas}

\phantomsection
\refstepcounter{subsection}
\addcontentsline{toc}{subsection}{\numberline{\thesubsection}Alexey Kutepov}

\persona{Alexey Kutepov (Tsoding)}
{images/kutepov.jpg}
{35 anni, nato in Russia, attualmente nomade digitale}
{Nomade digitale - Attualmente in Europa dell'Est}
{Laurea in Computer Science conseguita in Russia nel 2012, con focus su sistemi operativi e programmazione a basso livello}
{Software Engineer e Content Creator dal 2016. Streamer di programmazione su Twitch e YouTube, specializzato in C, Rust, e sviluppo di linguaggi di programmazione. Creatore di progetti open source e tool di sviluppo. Conosciuto per il suo approccio diretto e senza filtri alla programmazione}
{Pragmatico e diretto, con approccio no-nonsense al codice. Perfezionista tecnico che valorizza semplicità ed efficienza. Ironico e sarcastico, ma genuinamente appassionato di condivisione della conoscenza tecnica}
{Appassionato di matematica e teoria dei linguaggi di programmazione. Interessato a minimalismo nel design software e filosofia Unix. Nel tempo libero esplora nuove tecnologie, contribuisce a progetti open source e sperimenta con hardware vintage}

\biografia{Alexey Kutepov, 35 anni, conosciuto online come Tsoding, è un software engineer e content creator nomade specializzato in programmazione di sistema e sviluppo di linguaggi. Dal 2016 condivide il suo processo creativo attraverso live coding su Twitch, dove programma in C e Rust senza script né editing.

Con un approccio pragmatico e diretto, Tsoding è diventato una figura di riferimento per developer che apprezzano la programmazione a basso livello e la filosofia del codice minimalista ed efficiente.}

\scopiobiettivi{
  \begin{itemize}
    \item \textbf{Tracciare bug nei progetti open source personali:} Gestire efficacemente gli issue dei suoi numerosi repository GitHub, mantenendo organizzazione chiara per la community di contributor.
    \item \textbf{Riprodurre e documentare bug complessi:} Creare report tecnici dettagliati con stack trace, memory dumps e condizioni di riproduzione per problemi di basso livello.
    \item \textbf{Prioritizzare fix in base a impatto tecnico:} Categorizzare issue distinguendo tra crash critici, memory leaks, problemi di performance e feature requests.
    \item \textbf{Integrare bug tracking nel workflow di live coding:} Utilizzare l'applicazione durante le sessioni di streaming per mostrare alla community come affrontare sistematicamente i problemi tecnici.
  \end{itemize}
}

\phantomsection
\refstepcounter{subsection}
\addcontentsline{toc}{subsection}{\numberline{\thesubsection}Hideo Kojima}

\vspace{1cm}

\persona{Hideo Kojima}
{images/hideo.png}
{61 anni, nato a Tokyo}
{Tokyo, Giappone - Quartiere Shibuya}
{Laurea in Economia conseguita presso l'Università di Tokyo nel 1986, con studi complementari in cinema e letteratura}
{Game Director, Producer e Autore dal 1986. Fondatore di Kojima Productions nel 2015. Creatore di franchise iconici come Metal Gear e Death Stranding. Visionario dell'industria videoludica con approccio cinematografico}
{Visionario e meticoloso, con attenzione maniacale ai dettagli narrativi e tecnici. Perfezionista, esigente ma collaborativo. Comunicatore carismatico con forte presenza mediatica}
{Cinefilo appassionato che colleziona film e oggetti da collezione. Lettore vorace di letteratura e manga. Interessato a tecnologia, intelligenza artificiale e futuro dell'intrattenimento digitale. Attivo sui social media dove condivide le sue passioni quotidiane}

\biografia{Hideo Kojima, 61 anni, è uno dei game director più influenti e riconosciuti a livello mondiale. Con una formazione in economia e una passione viscerale per il cinema, ha rivoluzionato l'industria videoludica dal 1986, creando opere che fondono gameplay innovativo e narrativa cinematografica.

Fondatore di Kojima Productions, è celebre per la sua attenzione maniacale ai dettagli e per la sua visione autoriale che ha ridefinito i confini tra videogiochi, cinema e arte interattiva.}

\scopiobiettivi{
  \begin{itemize}
    \item \textbf{Mantenere la visione artistica del progetto:} Assicurarsi che ogni bug risolto non comprometta l'esperienza narrativa e l'immersione cinematografica prevista per il gioco.
    \item \textbf{Coordinare team multidisciplinari internazionali:} Utilizzare l'applicazione per sincronizzare il lavoro tra programmatori, designer, artisti e tester distribuiti in diverse sedi globali.
    \item \textbf{Prioritizzare issue che impattano l'esperienza emotiva:} Categorizzare i bug in base al loro effetto sulla narrazione, l'atmosfera e il coinvolgimento emotivo del giocatore.
    \item \textbf{Documentare dettagliatamente ogni problema:} Creare report completi con context narrativo, screenshot, video e note precise per guidare il team verso soluzioni che rispettino la visione creativa.
    \item \textbf{Monitorare milestone critiche pre-release:} Tenere traccia dello stato di risoluzione degli issue in vista di demo, eventi stampa e lancio finale, garantendo standard qualitativi eccellenti.
  \end{itemize}
}

\phantomsection
\refstepcounter{subsection}
\addcontentsline{toc}{subsection}{\numberline{\thesubsection}Michele Deserto}

\vspace{1cm}

\persona{Michele Deserto}
{images/sabaku.png}
{35 anni, nato a Bari, residente a Milano da 8 anni}
{Milano, Italia - Zona Porta Nuova}
{Diploma in Architettura Moderna conseguito a Bari nel 2009, con specializzazione in design sostenibile}
{Content Creator dal 2015, specializzato in recensioni videoludiche e tecnologia su YouTube e Twitch. Appassionato di giochi indie, RPG e hardware. Assunto da poco come bug tester dalla FromSoftware}
{Flemmatico e riflessivo, con approccio creativo ai problemi. Calmo, empatico e buon ascoltatore con la sua community}
{Oltre ai contenuti digitali, ama viaggiare scoprendo nuove culture e tradizioni. Appassionato di architettura e fotografia urbana. Nel tempo libero cucina e suona la chitarra}

\biografia{Michele Deserto, 35 anni, è un content creator milanese specializzato in videogiochi e tecnologia. Con un diploma in architettura moderna e un approccio flemmatico e creativo, dal 2015 produce contenuti su YouTube e Twitch dedicati a giochi indie, RPG e innovazioni tech.

Appassionato viaggiatore, ama scoprire nuove culture e tradizioni, che spesso ispirano i suoi contenuti digitali.}

\scopiobiettivi{
  \begin{itemize}
    \item \textbf{Diminuire la presenza di bug nei giochi recensiti:} Garantire che i giochi presentino meno bug tecnici, migliorando l'esperienza di gioco per i suoi spettatori.
    \item \textbf{Tracciare efficacemente i bug critici:} Utilizzare l'applicazione per categorizzare e prioritizzare gli issue in base alla loro gravità e impatto sul gameplay.
    \item \textbf{Collaborare con il team di sviluppo:} Condividere report dettagliati e screenshot attraverso l'applicazione per facilitare la comunicazione con i programmatori.
    \item \textbf{Monitorare lo stato di risoluzione:} Tenere traccia dei bug segnalati e verificare quali vengono risolti nelle patch successive.
    \item \textbf{Organizzare i test per piattaforme diverse:} Gestire gli issue separando i bug per console, PC e altre piattaforme su cui vengono testati i giochi.
  \end{itemize}
}

\phantomsection
\refstepcounter{subsection}
\addcontentsline{toc}{subsection}{\numberline{\thesubsection}Yan Chernikov}

\vspace{1cm}

\persona{Yan Chernikov (TheCherno)}
{images/thecherno.jpg}
{29 anni, nato in Australia, residente negli Stati Uniti}
{Los Angeles, California - USA}
{Laurea in Computer Science conseguita in Australia nel 2017, con specializzazione in computer graphics e game engine development}
{Software Engineer e Content Creator dal 2012. Creatore della popolare serie YouTube su C++ e game engine development. Lead Engine Programmer con esperienza in AAA studios. Fondatore di Hazel Engine, un game engine educativo open source}
{Metodico e didattico, con approccio sistematico all'insegnamento della programmazione. Paziente e chiaro nelle spiegazioni, ma esigente sulla qualità del codice. Entusiasta e motivante con la sua community}
{Appassionato di architettura software e design patterns. Interessato a rendering graphics, fisica e ottimizzazione performance. Nel tempo libero sperimenta con nuove tecnologie grafiche, contribuisce alla community open source e gioca a giochi competitivi}

\biografia{Yan Chernikov, 29 anni, conosciuto come TheCherno, è un software engineer e educator specializzato in C++ e game engine development. Dal 2012 produce contenuti educativi di alta qualità su YouTube, rendendo accessibili concetti complessi di programmazione grafica e architettura di game engine.

Con esperienza in studi AAA e una passione genuina per l'insegnamento, TheCherno ha costruito una delle community più rispettate per developer che vogliono comprendere profondamente come funzionano i motori grafici moderni.}

\scopiobiettivi{
  \begin{itemize}
    \item \textbf{Gestire issue del progetto Hazel Engine:} Organizzare e prioritizzare bug report e feature requests della community per il suo game engine educativo open source.
    \item \textbf{Documentare problemi di rendering e graphics:} Creare report dettagliati per bug complessi relativi a shader, pipeline grafiche e ottimizzazioni rendering con screenshot e frame analysis.
    \item \textbf{Coordinare sviluppo tra tutorial series:} Utilizzare l'applicazione per pianificare quali issue affrontare durante le prossime puntate della serie YouTube, mantenendo coerenza didattica.
    \item \textbf{Categorizzare per subsystem dell'engine:} Organizzare issue per moduli specifici (renderer, physics, ECS, scripting) facilitando la navigazione per contributor e studenti.
    \item \textbf{Tracciare problemi cross-platform:} Monitorare bug specifici per Windows, macOS e Linux, assicurando che Hazel Engine funzioni correttamente su tutte le piattaforme.
  \end{itemize}
}